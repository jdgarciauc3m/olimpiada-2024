\subsection{Conceptos}

\begin{frame}[t,fragile]{Programación genérica}
\begin{itemize}
  \item La programación genérica permite definir una función
        (o un tipo) para una familia de tipos.
\begin{lstlisting}
auto cuadrado(auto x) {
  return x*x;
}

void f() {
  auto a = cuadrado(2.0); // cuadrado<double>(2.0)
  auto b = cuadrado(2); // cuadrado<int>(2)
  auto c = cuadrado(2L); // cuadrado<long>(2L)
  auto d = cuadrado("hola"); // Error: "hola" * "hola"
  //...
}
\end{lstlisting}
\end{itemize}
\end{frame}

\begin{frame}[t,fragile]{Restricciones sobre tipos}
\begin{itemize}
  \item Se puede restringir un paráemtro genérico solamente
        a los tipos que cumplen una propiedad.
\begin{lstlisting}
auto cuadrado(std::integral auto x) {
  return x*x;
}

void f() {
  auto a = cuadrado(2.0); // Error: double no es integral
  auto b = cuadrado(2); // OK: cuadrado<int>(2)
  auto c = cuadrado(2L); // OK: cuadrado<long>(2L)
  auto d = cuadrado("hola"); // Error: "hola" no es integral
  //...
}
\end{lstlisting}
\end{itemize}
\end{frame}

\begin{frame}[t,fragile]{Definición de nuevos conceptos}
\begin{itemize}
  \item Se pueden combinar conceptos existentes para definir nuevos conceptos.
\begin{lstlisting}
template <typename T>
concept numerico = std::integral<T> or std::floating_point<T>;

auto cuadrado(numerico auto x) {
  return x*x;
}

void f() {
  auto a = cuadrado(2.0); // OK: cuadrado<double>
  auto b = cuadrado(2); // OK: cuadrado<int>(2)
  auto c = cuadrado(2L); // OK: cuadrado<long>(2L)
  auto d = cuadrado("hola"); // Error: "hola" no es numerico
  //...
}
\end{lstlisting}
\end{itemize}
\end{frame}
