\subsection{Retorno múltiple de funciones}

\begin{frame}[t,fragile]{Devolución múltiple}
\begin{itemize}
  \item ¿Puedo devolver más de un valor de una función?\pause
\begin{lstlisting}
std::tuple<int,int> divide(int dividendo, int divisor) {
  return std::tuple {
    dividendo / divisor,
    dividendo % divisor
  };
}
\end{lstlisting}
\end{itemize}

\begin{columns}[T]

\column{.1\textwidth}
\column{.5\textwidth}

\begin{itemize}
  \pause
  \item ¿Cómo lo uso (anticuado)?
\begin{lstlisting}
void f() {
  auto resultado = divide(10, 3);
  int cociente = std::get<0>(resultado);
  int resto = std::get<1>(resultado);
  //...
\end{lstlisting}
\end{itemize}
\column{.5\textwidth}

\begin{itemize}
  \pause
  \item ¿Cómo lo uso (moderno)?
\begin{lstlisting}
void f() {
  auto [cociente, resto] = divide(10, 3);
  //...
\end{lstlisting}
\end{itemize}

\end{columns}
\end{frame}

\begin{frame}[t,fragile]{Combinando ideas}
\begin{itemize}
  \item Una vinculación estructurada se puede combinar con un bucle
        basado en rango.
\begin{lstlisting}
void imprime_notas(const std::map<std::string, double> & notas) {
  for (auto & [nombre, nota] : notas) {
    std::cout << nombre << ": " << nota << '\n';
  }
}
\end{lstlisting}
\end{itemize}
\end{frame}
