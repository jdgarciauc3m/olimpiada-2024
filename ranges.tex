\subsection{Procesamiento de rangos}

\begin{frame}[t,fragile]{¿Qué es un rango?}
\begin{itemize}
  \item Tradicionalmente los algoritmos de la biblioteca estándar
        definidos en términos de iteradores.
\begin{lstlisting}
void imprime_en_orden(const std::vector<int> & v) {
  std::vector<int> tmp = v;
  std::sort(tmp.begin(), tmp.end());
  for (auto x : tmp) { std::cout << x << " "; }
}
\end{lstlisting}

  \vfill\pause
  \item Se introduce la idea de \textmark{rango} o secuencia de valores.
\begin{lstlisting}
void imprime_en_orden(const std::vector<int> & v) {
  std::vector<int> tmp = v;
  std::ranges::sort(tmp);
  for (auto x : tmp) { std::cout << x << " "; }
}
\end{lstlisting}
\end{itemize}
\end{frame}

\begin{frame}[t,fragile]{Vistas sobre una secuencia}
\begin{itemize}
  \item Una \textmark{vista} es un rango ligero que referencia a datos
        de otro rango.
\begin{lstlisting}
std::vector<int> v {1,2,3,4,5,6,7,8,9};

auto w = v
         | std::views::filter([](auto x) { return x % 2 == 0; })
         | std::views::transform([](auto x) { return x*x; })
         | std::views::take(3);

for (auto x : w) { std::cout << x << ' '; } // 4, 16, 36
\end{lstlisting}
\end{itemize}
\end{frame}
