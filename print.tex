\subsection{Salida de texto}

\begin{frame}[t,fragile]{Hola mundo a lo largo del tiempo}
\begin{itemize}
  \item El ejemplo clásico
\begin{lstlisting}
int main() {
  std::cout << "Hola mundo" << std::endl;
}
\end{lstlisting}

  \pause
  \item Aunque mejor no abusar de \cppid{std::endl}.
\begin{lstlisting}
int main() {
  std::cout << "Hola mundo\n";
}
\end{lstlisting}

  \pause
  \item Una forma más moderna de imprimir
\begin{lstlisting}
int main() {
  std::print("Hola mundo");
}
\end{lstlisting}
\end{itemize}
\end{frame}

\begin{frame}[t,fragile]{Impresión de valores}
\begin{itemize}
\item Uso del operador \cppkey{<{}<}.
\begin{lstlisting}
void imprime(std::string_view nombre, int edad, double altura) {
  std::cout << "Hola " << nombre << ". Tienes " << edad 
            << " años y mides " << altura << "m.\n";
}
\end{lstlisting}

\pause
\item Uso del operador \cppkey{<{}<}.
\begin{lstlisting}
void imprime(std::string_view nombre, int edad, double altura) {
  std::print("Hola {}. Tienes {} años y mides {} m.",
             nombre, edad, altura);
}
\end{lstlisting}

\end{itemize}
\end{frame}
