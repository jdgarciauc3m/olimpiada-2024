\subsection{Bucles basados en rango}

\begin{frame}[t,fragile]{Recorrido de un vector}
\vspace{-1em}
\begin{itemize}
\item Recorrido basado en índice.
\begin{lstlisting}[escapechar=@]
void imprime_valores(const std::vector<double> & v) {
  for (int i=0; i<@\color{red}=@std::ssize(v); ++i) {
    std::cout << i << ' ';
  }
}
\end{lstlisting}

\vfill\pause
\item Recorrido basado en iteradores.
\begin{lstlisting}
void imprime_valores(const std::vector<double> & v) {
  for (std::vector<double>::const_iterator i=v.begin(); i!=v.end(); ++i) {
    std::cout << *i << ' ';
  }
}
\end{lstlisting}
\vfill\pause
\item Recorrido basado en rango.
\begin{lstlisting}
void imprime_valores(const std::vector<double> & v) {
  for (auto x : v) { std::cout << x << ' '; }
}
\end{lstlisting}
\end{itemize}
\end{frame}

\begin{frame}[t,fragile]{Otros usos}
\begin{itemize}
  \item Recorrido de una lista de valores.
\begin{lstlisting}
void imprimie_impares() {
  for (auto x : {1, 3, 5, 7, 9}) {
    std::cout << x << ' ';
  }
}
\end{lstlisting}

  \vfill\pause
  \item Recorrido de una cadena
\begin{lstlisting}
std::string elimina_vocales(std::string_view s) {
  std::string resultado;
  for (auto c : s) { 
    if (!es_vocal(c)) {
      resultado.push_back(c);
    }
  }
  return resultado;
}
\end{lstlisting}
\end{itemize}
\end{frame}
