\subsection{Módulos}

\begin{frame}[t,fragile]{Un cambio fundamental en el modelo de compilación}
\begin{itemize}
  \item El modelo basado en directivas \cppkey{include} se basa en la
        idea de sustitución de texto.
    \begin{itemize}
      \item Cada vez que se encuentra una inclusión se sustituye esa línea
            por el contenido del archivo.
      \item Es un modelo que ralentiza las compilaciones.
      \item Complica la estructura del software.
    \end{itemize}
\end{itemize}

\begin{columns}[T]

\column{.5\textwidth}
\begin{lstlisting}
#include <iostream>

int main() {
  std::println("¡Hola mundo!");
}
\end{lstlisting}

\pause
\column{.5\textwidth}
\begin{lstlisting}
import std;

int main() {
  std::println("¡Hola mundo!");
}
\end{lstlisting}

\end{columns}
\end{frame}

\begin{frame}[t,fragile]{Módulos en acción}
\begin{itemize}
  \item Los módulos permiten remplazar el enfoque basado en el preprocesador.
    \begin{itemize}
      \item Un programa está formado por unidades de módulos.
      \item Cada módulo importa y exporta entidades.
    \end{itemize}
\end{itemize}

\begin{columns}[T]

\column{.5\textwidth}
\begin{lstlisting}
export module aritmetica;

export double suma(double x, double y) {
  return x+y;
}
\end{lstlisting}

\column{.5\textwidth}
\begin{lstlisting}
import aritmetica;
import std;

int main() {
  std::println("resultado = {}", suma(2,3));
}
\end{lstlisting}

\begin{lstlisting}
\end{lstlisting}

\end{columns}
\end{frame}
